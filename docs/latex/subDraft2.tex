\documentclass[12pt,a4paper]{article} 
\usepackage[latin1]{inputenc}
\usepackage{pdfpages}
\usepackage{graphicx}
\usepackage{float}
\usepackage{amsmath} 
\usepackage{amsfonts} 
\usepackage{amssymb}
\usepackage{color}
\usepackage{multicol}
\usepackage{multirow}
\usepackage{makeidx} 
\usepackage[english]{babel} 
\usepackage[hidelinks]{hyperref} 
\usepackage{array}
\usepackage{blindtext}
\usepackage{setspace}
\usepackage{parskip}

\linespread{1.1}
\begin{document} 
\definecolor{PSU}{RGB}{75,125,50} 
\begin{titlepage} 
\begin{center} 
 
% leave tilde after graphic, it designates par format, needed for formating 
\includegraphics[width=.75\textwidth]{./PSU_logo.png}~\\[.5cm] 
 
\textsc{\LARGE \color{PSU} Maseeh College of Engineering}\\[1.5cm] 
 
\textsc{\Large Prject Proposal}\\[0.5cm]
\textsc{\Large Submission Draft \#2}\\[0.5cm]
\vspace{1cm} 
% Title 
 
{ \huge \bfseries\color{PSU} A Block of Code\\[0.4cm] } 
  \large Senior Capstone Project
 
% \hrule  
\vspace{2.5cm} 
% Team Members 
 \begin{multicols}{2}
    
\begin{flushleft}
\noindent 
 \large 
\emph{\color{PSU}Team Members:}\\ 
Nathan \textsc{Bryant}\\ 
Daniel \textsc{Frister}\\
Tyler  \textsc{Hart}\\
Jacob   \textsc{Micikiewicz}\\
Greg    \textsc{Stromire}\\
\end{flushleft} 

 \begin{flushleft}
  \large 
 \emph{\color{PSU}Erebus Labs:} \\ 
 Dr. Mike  \textsc{Borowczak}\\ 
 \emph{\color{PSU}University of Wyoming}\\
 DR. Andrea \textsc{Burrows}\\ 
 \emph{\color{PSU}PSU Advisor:}\\ 
 Roy \textsc{Kravitz} 
 \end{flushleft}


 \end{multicols}` 
\vfill 
 
% Bottom of the page 
{\large \today} 
 
\end{center} 
\end{titlepage} 
 \tableofcontents

\section{Needs Statement}
The skills involved in programming encompass a large cross disciplinary curriculum. Learning to write code is easy, however imparting the knowledge and intuition necessary to be  skilled programmer is no mean feat. Under the current educational system economics weigh heavily on a schools ability to provide their student opportunities that are not officially budgeted. Worse still the economic divide is growing to encompass access to technology. There needs to be a low cost method of teaching advanced coding skills in school were access to a computer may be limited at best.  
\section{Objective}
Creating a tactile learning tool to instruct primary school students in the basics of computer programming is the primary goal of this venture. This tool will allow students to arrange blocks to perform basic operations and assignments. Verification of valid code structures will provide feedback for learning, and control of simple outputs connected to the system can give students a goal for their project.

\section{Background}

Currently there are is a limited selection of methods to teach younger groups of students about computer programming, and those that do exist rely on the use of software. This project is aimed to produce a learning aid that will function in a setting where there are not enough computers available for student to use. Basic functionality of the system should not require the presence of a computer.
The use of a visual-tactile system to teach new skills to children has a long and well established history. In terms of this project the system can be thought of as a, \textit{Spielgaben}, a subject specific learning module. The purpose of which is to allow students of any age to be introduced to the subject using a familiar process that promotes creativity. Allowing the module to be untethered, separate from a computer, will also allow the teaching of programming without the other distractions available on a PC, and give students that have trouble with abstraction a set of objects to focus on.

\section{Marketing Requirements}
The final package will be a set of between twenty to thirty blocks typically about two inches along the longest edge. The set will consist of a central processor block, a small number of dedicated output blocks, and the function and operation blocks used to implement the code. Power will be distributed through the network of connected blocks starting at a battery located near/with the processor, and all of the power and data connections for the system will be through contact pads on the edges of each block. These contact pads will be held in place during use by a coupling mechanism which holds adjacent blocks together. Error notifications will be available to indicate faults in the code at the individual block.

\section{Risk Management}
\subsection{Defining Risks}
 An assessment of project specific risks addressed by potential detriments, breaks risks down by project phase, and then by weight under each phase.
 
 
 \hspace{.35cm} \textbf{Risk types:}
 \begin{enumerate}
 \item \textbf{Scope}; risks of scope are those that caused the project to be delayed, too complex, or trivial. Frequently these are cause by poorly defined project specifications, attempting to do much, or becoming unnecessarily focused on specific details.
 \item \textbf{Conceptual}; risks belonging here are the products of some failure in understanding a problem or it's intended solution. One example could be  attempting to predict position by triangulation but forgetting to consider velocity. 
 \item \textbf{Physical}; risks of this type include problems that manifest physically, most commonly refereed to as \textit{bugs}(software bugs are included here. )
 \item \textbf{Incalculable}; risks of this nature include random and unforeseeable events. Shipment delays, inclement weather, and distributor component substitution are all examples of such risks.
 \end{enumerate}
 \subsection{Mitigating Risk}
\end{document} 
